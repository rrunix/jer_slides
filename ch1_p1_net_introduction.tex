\documentclass[ignorenonframetext,xcolor=dvipsnames,]{beamer}
\setbeamertemplate{caption}[numbered]
\setbeamertemplate{caption label separator}{: }
\setbeamercolor{caption name}{fg=normal text.fg}
%\beamertemplatenavigationsymbolsempty
%\usepackage{amssymb,amsmath}
\usepackage{lmodern}
\usepackage{ifxetex,ifluatex}
\usepackage{fixltx2e} % provides \textsubscript
\ifnum 0\ifxetex 1\fi\ifluatex 1\fi=0 % if pdftex
  \usepackage[T1]{fontenc}
  \usepackage[utf8]{inputenc}
%%\else % if luatex or xelatex
%  \ifxetex
%    \usepackage{mathspec}
%  \else
%    \usepackage{fontspec}
%  \fi
%  \defaultfontfeatures{Ligatures=TeX,Scale=MatchLowercase}
\fi
% use upquote if available, for straight quotes in verbatim environments
\IfFileExists{upquote.sty}{\usepackage{upquote}}{}
% use microtype if available
\IfFileExists{microtype.sty}{%
\usepackage{microtype}
\UseMicrotypeSet[protrusion]{basicmath} % disable protrusion for tt fonts
}{}
\newif\ifbibliography
\hypersetup{
            pdftitle={Introducción a las Redes de Ordenadores},
            pdfauthor={Alberto Fernández de Isabel; Rubén Rodríguez Fernández (@rrunix)},
            pdfborder={0 0 0},
            breaklinks=true}
\usepackage{color}
\usepackage{fancyvrb}
\newcommand{\VerbBar}{|}
\newcommand{\VERB}{\Verb[commandchars=\\\{\}]}
\DefineVerbatimEnvironment{Highlighting}{Verbatim}{commandchars=\\\{\}}
% Add ',fontsize=\small' for more characters per line
\usepackage{framed}
\definecolor{shadecolor}{RGB}{241,243,245}
\newenvironment{Shaded}{\begin{snugshade}}{\end{snugshade}}
\newcommand{\AlertTok}[1]{\textcolor[rgb]{0.68,0.00,0.00}{#1}}
\newcommand{\AnnotationTok}[1]{\textcolor[rgb]{0.37,0.37,0.37}{#1}}
\newcommand{\AttributeTok}[1]{\textcolor[rgb]{0.40,0.45,0.13}{#1}}
\newcommand{\BaseNTok}[1]{\textcolor[rgb]{0.68,0.00,0.00}{#1}}
\newcommand{\BuiltInTok}[1]{\textcolor[rgb]{0.00,0.23,0.31}{#1}}
\newcommand{\CharTok}[1]{\textcolor[rgb]{0.13,0.47,0.30}{#1}}
\newcommand{\CommentTok}[1]{\textcolor[rgb]{0.37,0.37,0.37}{#1}}
\newcommand{\CommentVarTok}[1]{\textcolor[rgb]{0.37,0.37,0.37}{\textit{#1}}}
\newcommand{\ConstantTok}[1]{\textcolor[rgb]{0.56,0.35,0.01}{#1}}
\newcommand{\ControlFlowTok}[1]{\textcolor[rgb]{0.00,0.23,0.31}{\textbf{#1}}}
\newcommand{\DataTypeTok}[1]{\textcolor[rgb]{0.68,0.00,0.00}{#1}}
\newcommand{\DecValTok}[1]{\textcolor[rgb]{0.68,0.00,0.00}{#1}}
\newcommand{\DocumentationTok}[1]{\textcolor[rgb]{0.37,0.37,0.37}{\textit{#1}}}
\newcommand{\ErrorTok}[1]{\textcolor[rgb]{0.68,0.00,0.00}{#1}}
\newcommand{\ExtensionTok}[1]{\textcolor[rgb]{0.00,0.23,0.31}{#1}}
\newcommand{\FloatTok}[1]{\textcolor[rgb]{0.68,0.00,0.00}{#1}}
\newcommand{\FunctionTok}[1]{\textcolor[rgb]{0.28,0.35,0.67}{#1}}
\newcommand{\ImportTok}[1]{\textcolor[rgb]{0.00,0.46,0.62}{#1}}
\newcommand{\InformationTok}[1]{\textcolor[rgb]{0.37,0.37,0.37}{#1}}
\newcommand{\KeywordTok}[1]{\textcolor[rgb]{0.00,0.23,0.31}{\textbf{#1}}}
\newcommand{\NormalTok}[1]{\textcolor[rgb]{0.00,0.23,0.31}{#1}}
\newcommand{\OperatorTok}[1]{\textcolor[rgb]{0.37,0.37,0.37}{#1}}
\newcommand{\OtherTok}[1]{\textcolor[rgb]{0.00,0.23,0.31}{#1}}
\newcommand{\PreprocessorTok}[1]{\textcolor[rgb]{0.68,0.00,0.00}{#1}}
\newcommand{\RegionMarkerTok}[1]{\textcolor[rgb]{0.00,0.23,0.31}{#1}}
\newcommand{\SpecialCharTok}[1]{\textcolor[rgb]{0.37,0.37,0.37}{#1}}
\newcommand{\SpecialStringTok}[1]{\textcolor[rgb]{0.13,0.47,0.30}{#1}}
\newcommand{\StringTok}[1]{\textcolor[rgb]{0.13,0.47,0.30}{#1}}
\newcommand{\VariableTok}[1]{\textcolor[rgb]{0.07,0.07,0.07}{#1}}
\newcommand{\VerbatimStringTok}[1]{\textcolor[rgb]{0.13,0.47,0.30}{#1}}
\newcommand{\WarningTok}[1]{\textcolor[rgb]{0.37,0.37,0.37}{\textit{#1}}}
\usepackage{longtable,booktabs}
\usepackage{caption}
% These lines are needed to make table captions work with longtable:
\makeatletter
\def\fnum@table{\tablename~\thetable}
\makeatother
\usepackage{graphicx,grffile}
\makeatletter
\def\maxwidth{\ifdim\Gin@nat@width>\linewidth\linewidth\else\Gin@nat@width\fi}
\def\maxheight{\ifdim\Gin@nat@height>\textheight0.8\textheight\else\Gin@nat@height\fi}
\makeatother
% Scale images if necessary, so that they will not overflow the page
% margins by default, and it is still possible to overwrite the defaults
% using explicit options in \includegraphics[width, height, ...]{}
\setkeys{Gin}{width=\maxwidth,height=\maxheight,keepaspectratio}

% Prevent slide breaks in the middle of a paragraph:
\widowpenalties 1 10000
\raggedbottom

\AtBeginPart{
  \let\insertpartnumber\relax
  \let\partname\relax
  \frame{\partpage}
}
\AtBeginSection{
  \ifbibliography
  \else
    \let\insertsectionnumber\relax
    \let\sectionname\relax
    \frame{\sectionpage}
  \fi
}
\AtBeginSubsection{
  \let\insertsubsectionnumber\relax
  \let\subsectionname\relax
  \frame{\subsectionpage}
}

\setlength{\parindent}{0pt}
\setlength{\parskip}{6pt plus 2pt minus 1pt}
\setlength{\emergencystretch}{3em}  % prevent overfull lines
\providecommand{\tightlist}{%
  \setlength{\itemsep}{0pt}\setlength{\parskip}{0pt}}
\setcounter{secnumdepth}{0}
\providecommand{\pandocbounded}[1]{#1}
\makeatletter
\@ifpackageloaded{tcolorbox}{}{\usepackage[skins,breakable]{tcolorbox}}
\@ifpackageloaded{fontawesome5}{}{\usepackage{fontawesome5}}
\definecolor{quarto-callout-color}{HTML}{909090}
\definecolor{quarto-callout-note-color}{HTML}{0758E5}
\definecolor{quarto-callout-important-color}{HTML}{CC1914}
\definecolor{quarto-callout-warning-color}{HTML}{EB9113}
\definecolor{quarto-callout-tip-color}{HTML}{00A047}
\definecolor{quarto-callout-caution-color}{HTML}{FC5300}
\definecolor{quarto-callout-color-frame}{HTML}{acacac}
\definecolor{quarto-callout-note-color-frame}{HTML}{4582ec}
\definecolor{quarto-callout-important-color-frame}{HTML}{d9534f}
\definecolor{quarto-callout-warning-color-frame}{HTML}{f0ad4e}
\definecolor{quarto-callout-tip-color-frame}{HTML}{02b875}
\definecolor{quarto-callout-caution-color-frame}{HTML}{fd7e14}
\makeatother
\makeatletter
\@ifpackageloaded{caption}{}{\usepackage{caption}}
\AtBeginDocument{%
\ifdefined\contentsname
  \renewcommand*\contentsname{Table of contents}
\else
  \newcommand\contentsname{Table of contents}
\fi
\ifdefined\listfigurename
  \renewcommand*\listfigurename{List of Figures}
\else
  \newcommand\listfigurename{List of Figures}
\fi
\ifdefined\listtablename
  \renewcommand*\listtablename{List of Tables}
\else
  \newcommand\listtablename{List of Tables}
\fi
\ifdefined\figurename
  \renewcommand*\figurename{Figure}
\else
  \newcommand\figurename{Figure}
\fi
\ifdefined\tablename
  \renewcommand*\tablename{Table}
\else
  \newcommand\tablename{Table}
\fi
}
\@ifpackageloaded{float}{}{\usepackage{float}}
\floatstyle{ruled}
\@ifundefined{c@chapter}{\newfloat{codelisting}{h}{lop}}{\newfloat{codelisting}{h}{lop}[chapter]}
\floatname{codelisting}{Listing}
\newcommand*\listoflistings{\listof{codelisting}{List of Listings}}
\makeatother
\makeatletter
\makeatother
\makeatletter
\@ifpackageloaded{caption}{}{\usepackage{caption}}
\@ifpackageloaded{subcaption}{}{\usepackage{subcaption}}
\makeatother

\title{Introducción a las Redes de Ordenadores}
\subtitle{Fundamentos de Internet y Arquitecturas de Red}
\author{Alberto Fernández de Isabel \and Rubén Rodríguez Fernández
(@rrunix)}
\titlegraphic{\raisebox{0.5cm}{\includegraphics[width=2.75cm]{images/DSLab_logo_1.png}} \hspace{1cm}
              \raisebox{0.5cm}{\includegraphics[width=2.75cm]{images/URJClogo}} \hspace{1cm}}
\date{2025-10-22}
\usecolortheme{DSLAB}

\usetheme{Madrid}
\usefonttheme{professionalfonts}

\usepackage{multicol}
   \newcommand{\btwocol}{\begin{multicols}{2}}
   \newcommand{\etwocol}{\end{multicols}}

\begin{document}


\frame{\titlepage}

\section{Introducción}\label{introducciuxf3n}

\begin{frame}{¿Qué es Internet?}
\phantomsection\label{quuxe9-es-internet}
\begin{columns}[T]
\begin{column}{0.5\linewidth}
\textbf{Etimología}: ``Interconnected Networks''

\begin{itemize}
\tightlist
\item
  Red global de redes interconectadas
\item
  Sistema descentralizado
\item
  Múltiples capas jerárquicas
\end{itemize}
\end{column}

\begin{column}{0.5\linewidth}
\textbf{Características principales}:

\begin{itemize}
\tightlist
\item
  Arquitectura distribuida
\item
  Resiliencia a fallos
\item
  Escalabilidad natural
\item
  Sin control centralizado
\end{itemize}
\end{column}
\end{columns}

\includegraphics[width=14.2in,height=3.01in]{ch1_p1_net_introduction_files/figure-beamer/mermaid-figure-1.png}
\end{frame}

\begin{frame}{Jerarquía de Redes}
\phantomsection\label{jerarquuxeda-de-redes}
\pandocbounded{\includegraphics[keepaspectratio]{ch1_p1_net_introduction_files/figure-beamer/mermaid-figure-10.png}}

\begin{itemize}
\tightlist
\item
  \textbf{PAN}: Red personal entre dispositivos cercanos.
\item
  \textbf{LAN}: Red local de casa/oficina/edificio.
\item
  \textbf{WLAN}: LAN inalámbrica (Wi-Fi).
\item
  \textbf{CAN}: Red de campus - conecta múltiples LANs.
\item
  \textbf{MAN}: Red metropolitana - cubre una ciudad, incluye redes de
  ISP y móviles (4G/5G)
\item
  \textbf{WAN}: Red de área amplia - conecta ciudades o países.
\item
  \textbf{Internet}: Red global - interconexión de todas las WANs del
  mundo
\end{itemize}

\note{\begin{itemize}
\tightlist
\item
  PAN: Personal (smartwatch, móvil)
\item
  LAN: Local (hogar, oficina)
\item
  MAN: Metropolitana (ciudad)
\item
  WAN: Área amplia (países)
\item
  Internet: Red global
\end{itemize}}
\end{frame}

\begin{frame}{Ejemplo: Mensaje Madrid → Tokio}
\phantomsection\label{ejemplo-mensaje-madrid-tokio}
Smartphone María (WiFI) en Madrid -\textgreater{} Takeshi LAN en la
Universidad de Tokyo

\begin{enumerate}[<+->]
\tightlist
\item
  \textbf{Origen LAN Madrid}: Smartphone → Router WiFi
\item
  \textbf{Router local → MAN}: ISP local → MAN Madrid
\item
  \textbf{MAN → WAN nacional}: MAN Madrid → WAN España
\item
  \textbf{WAN → Internet global}: España → Backbone internacional
\item
  \textbf{Llegada a Japón}: WAN Japón → MAN Tokio
\item
  \textbf{MAN → CAN}: MAN Tokio → Universidad
\item
  \textbf{CAN → LAN}: Campus → LAN específica
\item
  \textbf{Destino final}: LAN → Dispositivo de Takeshi
\end{enumerate}
\end{frame}

\begin{frame}[fragile]{Un caso un poco más real}
\phantomsection\label{un-caso-un-poco-muxe1s-real}
\begin{itemize}
\tightlist
\item
  Probad a ejecutar en vuestras terminales
  \texttt{traceroute\ www.google.es} (\texttt{tracert\ www.google.es} en
  Windows)
\item
  ¿Qué información estáis obteniendo?
\item
  Comparadla con vuestros compañer@s. ¿Es la misma?
\end{itemize}
\end{frame}

\begin{frame}{Internet simplificado}
\phantomsection\label{internet-simplificado}
\pandocbounded{\includegraphics[keepaspectratio]{resources/small_network.png}}
\end{frame}

\begin{frame}{Componentes Clave}
\phantomsection\label{componentes-clave}
\begin{columns}[T]
\begin{column}{0.5\linewidth}
\begin{block}{Router}
\phantomsection\label{router}
\begin{itemize}
\tightlist
\item
  Conecta \textbf{diferentes redes}
\item
  Usa direcciones \textbf{IP}
\item
  Enrutamiento ``hop by hop''
\item
  Opera entre redes distantes
\end{itemize}
\end{block}
\end{column}

\begin{column}{0.5\linewidth}
\begin{block}{Switch}
\phantomsection\label{switch}
\begin{itemize}
\tightlist
\item
  Conecta dispositivos en \textbf{misma red}
\item
  Usa direcciones \textbf{MAC}
\item
  Entrega local inteligente
\item
  Opera dentro de la LAN
\end{itemize}
\end{block}
\end{column}
\end{columns}
\end{frame}

\begin{frame}{En nuestras casas}
\phantomsection\label{en-nuestras-casas}
Entonces\ldots{} ¿Esto que es?

\pandocbounded{\includegraphics[keepaspectratio]{resources/router_switch.png}}
\end{frame}

\begin{frame}[fragile]{Identificadores en Red}
\phantomsection\label{identificadores-en-red}
\begin{columns}[T]
\begin{column}{0.33\linewidth}
\begin{block}{Dirección IP}
\phantomsection\label{direcciuxf3n-ip}
\begin{itemize}
\tightlist
\item
  ``Dirección postal''
\item
  Localiza en la red
\item
  Ejemplo: 192.168.1.100
\item
  Puede cambiar
\end{itemize}
\end{block}
\end{column}

\begin{column}{0.33\linewidth}
\begin{block}{Dirección MAC}
\phantomsection\label{direcciuxf3n-mac}
\begin{itemize}
\tightlist
\item
  ``DNI del dispositivo''
\item
  Única y permanente
\item
  Asignada por fabricante
\item
  No cambia nunca
\end{itemize}
\end{block}
\end{column}

\begin{column}{0.33\linewidth}
\begin{block}{Protocolo ARP}
\phantomsection\label{protocolo-arp}
\begin{itemize}
\tightlist
\item
  ``Directorio telefónico''
\item
  Traduce IP ↔ MAC
\item
  Permite entrega final
\item
  Opera localmente
\end{itemize}
\end{block}
\end{column}
\end{columns}

\begin{tcolorbox}[enhanced jigsaw, opacitybacktitle=0.6, toptitle=1mm, bottomtitle=1mm, opacityback=0, left=2mm, colbacktitle=quarto-callout-note-color!10!white, breakable, coltitle=black, colback=white, bottomrule=.15mm, titlerule=0mm, leftrule=.75mm, toprule=.15mm, arc=.35mm, title=\textcolor{quarto-callout-note-color}{\faInfo}\hspace{0.5em}{Ejercicio}, rightrule=.15mm, colframe=quarto-callout-note-color-frame]

Prueba a ejecutar \texttt{ifconfig} en tu terminal MacOS/Linux o
\texttt{ipconfig} en Windows. ¿Qué ves?.

\end{tcolorbox}
\end{frame}

\begin{frame}{Protocolos de Red}
\phantomsection\label{protocolos-de-red}
\begin{quote}
\textbf{Protocolo}: Serie de pasos bien definidos que especifican cómo
intercambiar información entre dispositivos
\end{quote}

\begin{block}{Analogía del tráfico urbano}
\phantomsection\label{analoguxeda-del-truxe1fico-urbano}
\begin{itemize}
\tightlist
\item
  \textbf{Sin protocolos}: Caos total, pérdida de información
\item
  \textbf{Con protocolos}: Flujo ordenado, comunicación efectiva
\end{itemize}
\end{block}
\end{frame}

\section{Historia de Internet}\label{historia-de-internet}

\begin{frame}{Linea temporal de Internet.}
\phantomsection\label{linea-temporal-de-internet.}
\pandocbounded{\includegraphics[keepaspectratio]{ch1_p1_net_introduction_files/figure-beamer/mermaid-figure-9.png}}

\begin{itemize}
\tightlist
\item
  ARPANET se creo con fines militares
\item
  Se creo y publicó el primer sitio web (CERN)
\item
  Burbuja punto com
\item
  Creación de las Redes sociales.
\item
  Inteligencia Artificial Generativa.
\item
  De \textasciitilde4 dispositivos (1969) a \textgreater100.000 millones
  (2025) en 50 años.
\end{itemize}
\end{frame}

\section{Infraestructura de Red}\label{infraestructura-de-red}

\begin{frame}{Sistemas Terminales (End Systems)}
\phantomsection\label{sistemas-terminales-end-systems}
\begin{quote}
Hosts (End systems): Son los dispositivos que \textbf{usan} Internet
como PCs, smartphones, IoT, servidores. Ejecutan aplicaciones de red.
\end{quote}

\begin{columns}[T]
\begin{column}{0.5\linewidth}
\begin{block}{Clasificación}
\phantomsection\label{clasificaciuxf3n}
\begin{itemize}
\tightlist
\item
  \textbf{Clientes}: Solicitan servicios
\item
  \textbf{Servidores}: Proporcionan servicios
\item
  Roles dinámicos (P2P)
\end{itemize}
\end{block}
\end{column}

\begin{column}{0.5\linewidth}
\includegraphics[width=5.88in,height=2.58in]{ch1_p1_net_introduction_files/figure-beamer/mermaid-figure-8.png}

\includegraphics[width=7.14in,height=2in]{ch1_p1_net_introduction_files/figure-beamer/mermaid-figure-7.png}
\end{column}
\end{columns}
\end{frame}

\begin{frame}{Redes de Acceso}
\phantomsection\label{redes-de-acceso}
\begin{quote}
Redes de acceso: Es la red en la que se conectan los host con el router
de borde.
\end{quote}

\begin{columns}[T]
\begin{column}{0.5\linewidth}
\textbf{Tecnologías host → router}

\begin{itemize}
\tightlist
\item
  WiFi 6: 200-400 Mb/s
\item
  Ethernet: 10 Gb/s
\item
  4G LTE: 50/15 Mb/s
\item
  5G: 300/50 Mb/s
\end{itemize}
\end{column}

\begin{column}{0.5\linewidth}
\textbf{Características}

\begin{itemize}
\tightlist
\item
  Alcance limitado
\item
  Velocidades variables
\item
  Medios compartidos vs dedicados
\end{itemize}
\end{column}
\end{columns}
\end{frame}

\begin{frame}{Tecnologías WAN}
\phantomsection\label{tecnologuxedas-wan}
\begin{quote}
Router de borde: Router que conecta la red de acceso con el núcleo de la
red.
\end{quote}

\begin{block}{Tecnologías comunes:}
\phantomsection\label{tecnologuxedas-comunes}
\begin{longtable}[]{@{}lll@{}}
\toprule\noalign{}
Tecnología & Velocidad típica & Estado 2025 \\
\midrule\noalign{}
\endhead
DSL/VDSL & 50/15 Mb/s & En declive \\
Cable HFC & 300/30 Mb/s & Estable \\
FTTH PON & 1000/1000 Mb/s & En expansión \\
FTTH P2P & 10000/10000 Mb/s & Premium \\
Satelital & 100/20 Mb/s & Nicho \\
\bottomrule\noalign{}
\end{longtable}
\end{block}
\end{frame}

\begin{frame}{Núcleo de la Red: ISPs}
\phantomsection\label{nuxfacleo-de-la-red-isps}
\begin{quote}
ISP (Internet Service Providers): Son los componentes del núcleo de la
red y proporcionan interconexión entre diferentes redes.
\end{quote}

\includegraphics[width=11.31in,height=2.77in]{ch1_p1_net_introduction_files/figure-beamer/mermaid-figure-6.png}

\begin{columns}[T]
\begin{column}{0.33\linewidth}
\begin{block}{Tier 1}
\phantomsection\label{tier-1}
\begin{itemize}
\tightlist
\item
  Cobertura global
\item
  Peering gratuito
\item
  AT\&T, Telefónica
\item
  10-100 Gb/s
\end{itemize}
\end{block}
\end{column}

\begin{column}{0.33\linewidth}
\begin{block}{Tier 2}
\phantomsection\label{tier-2}
\begin{itemize}
\tightlist
\item
  Cobertura regional/nacional
\item
  Pagan tránsito a Tier 1
\item
  Peering selectivo
\end{itemize}
\end{block}
\end{column}

\begin{column}{0.33\linewidth}
\begin{block}{Tier 3}
\phantomsection\label{tier-3}
\begin{itemize}
\tightlist
\item
  Acceso local
\item
  Última milla
\item
  Usuarios finales
\item
  Sin peering
\end{itemize}
\end{block}
\end{column}
\end{columns}
\end{frame}

\section{Modelos de Referencia}\label{modelos-de-referencia}

\begin{frame}{Arquitecturas por Capas}
\phantomsection\label{arquitecturas-por-capas}
\begin{columns}[T]
\begin{column}{0.5\linewidth}
\begin{itemize}
\tightlist
\item
  Cada capa = responsabilidad específica
\item
  Servicios a capa superior
\item
  Usa servicios de capa inferior
\item
  Desarrollo independiente
\end{itemize}
\end{column}

\begin{column}{0.5\linewidth}
\includegraphics[width=2.17in,height=4.27in]{ch1_p1_net_introduction_files/figure-beamer/mermaid-figure-5.png}
\end{column}
\end{columns}
\end{frame}

\begin{frame}{Encapsulación}
\phantomsection\label{encapsulaciuxf3n}
\begin{columns}[T]
\begin{column}{0.35\linewidth}
\begin{itemize}
\tightlist
\item
  Cada capa añade headers
\item
  Datos superiores = payload
\item
  No modifica contenido interno
\end{itemize}
\end{column}

\begin{column}{0.65\linewidth}
\includegraphics[width=6.91in,height=3.7in]{ch1_p1_net_introduction_files/figure-beamer/mermaid-figure-4.png}
\end{column}
\end{columns}
\end{frame}

\begin{frame}{Encapsulación + Arquitectura por capas}
\phantomsection\label{encapsulaciuxf3n-arquitectura-por-capas}
\includegraphics[width=12.48in,height=3.49in]{ch1_p1_net_introduction_files/figure-beamer/mermaid-figure-3.png}

\includegraphics[width=12.48in,height=3.49in]{ch1_p1_net_introduction_files/figure-beamer/mermaid-figure-2.png}
\end{frame}

\begin{frame}{Modelos OSI vs TCP/IP}
\phantomsection\label{modelos-osi-vs-tcpip}
\begin{columns}[T]
\begin{column}{0.5\linewidth}
\pandocbounded{\includegraphics[keepaspectratio]{ch1_p1_net_introduction_files/mediabag/resources/tcp_osi.pdf}}
\end{column}

\begin{column}{0.5\linewidth}
\begin{itemize}
\tightlist
\item
  \textbf{OSI}: 7 capas, modelo teórico
\item
  \textbf{TCP/IP}: 4 capas, usado en Internet
\end{itemize}
\end{column}
\end{columns}

\begin{quote}
⚠️ \textbf{Nota}: TCP/IP no es un protocolo, hace referencia a una pila
de protocolos. Además, no tiene porque utilizar necesariamente TCP,
podría ser UDP.
\end{quote}

\note{TCP/IP es superior a OSI porque es simple, práctico y probado.
Mientras OSI tiene 7 capas teóricas diseñadas en comités académicos,
TCP/IP usa solo 4 capas que realmente se necesitan y ha funcionado en
Internet durante más de 50 años. TCP/IP es más eficiente (menos
overhead), más flexible (se adapta fácilmente), y tiene adopción masiva
con ecosistemas completos de hardware y software. OSI llegó tarde al
mercado cuando TCP/IP ya dominaba, y sus capas extra (como Sesión y
Presentación) añaden complejidad innecesaria para la mayoría de
aplicaciones reales. En resumen: OSI es el modelo perfecto para estudiar
redes, pero TCP/IP es lo que realmente mueve Internet.

\begin{itemize}
\tightlist
\item
  Rigid Foundation: X11's network-based protocol became unchangeable due
  to backward compatibility, preventing modern optimizations
\item
  Performance Bottlenecks: Every graphics operation had to go through
  multiple layers (app → toolkit → X11 → server), creating unnecessary
  overhead for local applications
\item
  Hardware Evolution Mismatch: When GPUs emerged, X11's
  software-rendering architecture couldn't efficiently utilize new
  hardware acceleration
\item
  Extension Hell: Adding modern features required bolt-on extensions
  (Composite, GLX, XRender) rather than clean integration, creating
  complexity
\item
  Layer Lock-In: Applications built on X11 assumptions were hard to
  migrate to better architectures, requiring complete rewrites (hence
  Wayland's slow adoption)
\end{itemize}}
\end{frame}

\begin{frame}{Nivel de Aplicación}
\phantomsection\label{nivel-de-aplicaciuxf3n}
\begin{quote}
Es el nivel en que desarrollamos aplicaciones.
\end{quote}

\begin{columns}[T]
\begin{column}{0.5\linewidth}
\begin{block}{OSI (Capas 7, 6, 5)}
\phantomsection\label{osi-capas-7-6-5}
\begin{itemize}
\tightlist
\item
  \textbf{Aplicación}: HTTP, FTP, DNS
\item
  \textbf{Presentación}: Cifrado, compresión
\item
  \textbf{Sesión}: Control de diálogos
\end{itemize}
\end{block}
\end{column}

\begin{column}{0.5\linewidth}
\begin{block}{TCP/IP}
\phantomsection\label{tcpip}
\begin{itemize}
\tightlist
\item
  Una sola capa integrada
\item
  Protocolos: HTTP/HTTPS, SMTP, FTP, DNS
\item
  Más práctico
\end{itemize}
\end{block}
\end{column}
\end{columns}
\end{frame}

\begin{frame}{Nivel de Transporte}
\phantomsection\label{nivel-de-transporte}
\begin{quote}
Gestiona la comunicación extremo a extremo entre aplicaciones.
\end{quote}

\begin{block}{Capa 4 (ambos modelos)}
\phantomsection\label{capa-4-ambos-modelos}
\begin{columns}[T]
\begin{column}{0.5\linewidth}
\textbf{TCP}

\begin{itemize}
\tightlist
\item
  Comunicación confiable
\item
  Control de flujo
\item
  Entrega ordenada
\item
  Corrección de errores
\end{itemize}
\end{column}

\begin{column}{0.5\linewidth}
\textbf{UDP}

\begin{itemize}
\tightlist
\item
  Comunicación rápida
\item
  Sin garantías
\item
  Ideal para tiempo real
\item
  Menor overhead
\end{itemize}
\end{column}
\end{columns}
\end{block}
\end{frame}

\begin{frame}[fragile]{Nivel de Red/Internet}
\phantomsection\label{nivel-de-redinternet}
\begin{quote}
Se encarga de encontrar el mejor camino para enviar datos a través de
múltiples redes. En otras palabras, se encarga del \textbf{enrutamiento}
de paquetes.
\end{quote}

\begin{block}{Capa 3 OSI / Capa Internet TCP/IP}
\phantomsection\label{capa-3-osi-capa-internet-tcpip}
\textbf{Protocolos}:

\begin{itemize}
\tightlist
\item
  IP: Protocolo principal
\item
  ICMP: Control y errores
\item
  ARP: Resolución de direcciones
\item
  OSPF, BGP: Protocolos de enrutamiento
\end{itemize}

\begin{tcolorbox}[enhanced jigsaw, opacitybacktitle=0.6, toptitle=1mm, bottomtitle=1mm, opacityback=0, left=2mm, colbacktitle=quarto-callout-note-color!10!white, breakable, coltitle=black, colback=white, bottomrule=.15mm, titlerule=0mm, leftrule=.75mm, toprule=.15mm, arc=.35mm, title=\textcolor{quarto-callout-note-color}{\faInfo}\hspace{0.5em}{Ejercicio}, rightrule=.15mm, colframe=quarto-callout-note-color-frame]

Prueba a ejecutar \texttt{ping\ www.google.es} en tu terminal. ¿Qué
ves?.

\end{tcolorbox}
\end{block}
\end{frame}

\begin{frame}{Nivel de Acceso Físico}
\phantomsection\label{nivel-de-acceso-fuxedsico}
\begin{quote}
Controla cómo los datos se transmiten físicamente a través del medio de
comunicación.
\end{quote}

\begin{columns}[T]
\begin{column}{0.5\linewidth}
\begin{block}{OSI (Capas 2 y 1)}
\phantomsection\label{osi-capas-2-y-1}
\begin{itemize}
\tightlist
\item
  \textbf{Enlace}: Control de errores, MAC
\item
  \textbf{Física}: Señales, voltajes
\end{itemize}
\end{block}
\end{column}

\begin{column}{0.5\linewidth}
\begin{block}{TCP/IP}
\phantomsection\label{tcpip-1}
\begin{itemize}
\tightlist
\item
  Capa de Acceso a Red
\item
  Combina ambas funciones
\item
  Ethernet, WiFi, etc.
\end{itemize}
\end{block}
\end{column}
\end{columns}
\end{frame}

\section{Rendimiento en Redes}\label{rendimiento-en-redes}

\begin{frame}{Métricas Principales}
\phantomsection\label{muxe9tricas-principales}
\begin{columns}[T]
\begin{column}{0.5\linewidth}
\begin{block}{Latencia}
\phantomsection\label{latencia}
\begin{itemize}
\tightlist
\item
  Tiempo que tarda un paquete en llegar a su destino.
\item
  ``Velocidad del vehículo''
\item
  Medida en ms
\end{itemize}
\end{block}
\end{column}

\begin{column}{0.5\linewidth}
\begin{block}{Throughput (Tasa de Transferencia Efectiva)}
\phantomsection\label{throughput-tasa-de-transferencia-efectiva}
\begin{itemize}
\tightlist
\item
  Datos enviados por cantidad de tiempo.
\item
  ``Número de carriles''
\item
  Medido en Mb/s o Gb/s
\end{itemize}
\end{block}
\end{column}
\end{columns}

\begin{quote}
⚠️ \textbf{Nota}: 1 MB/s = 8 Mb/s
\end{quote}
\end{frame}

\begin{frame}{Throughput vs Bandwidth}
\phantomsection\label{throughput-vs-bandwidth}
\begin{columns}[T]
\begin{column}{0.5\linewidth}
\begin{block}{Bandwidth}
\phantomsection\label{bandwidth}
\begin{itemize}
\tightlist
\item
  Capacidad \textbf{máxima teórica}
\item
  Límite físico del canal
\item
  Condiciones ideales
\end{itemize}
\end{block}
\end{column}

\begin{column}{0.5\linewidth}
\begin{block}{Throughput}
\phantomsection\label{throughput}
\begin{itemize}
\tightlist
\item
  Transferencia \textbf{real}
\item
  Limitado por el componente más lento
\item
  Condiciones reales
\end{itemize}
\end{block}
\end{column}
\end{columns}
\end{frame}

\begin{frame}[fragile]{Latencia vs Throughput}
\phantomsection\label{latencia-vs-throughput}
Comparativa del efecto de la latencia y throughput en el tiempo para
enviar una cantidad de datos.

\begin{Shaded}
\begin{Highlighting}[]
\NormalTok{// Sliders agrupados en un formulario}
\NormalTok{viewof settings = Inputs.form(\{}
\NormalTok{  "Tamaño total": Inputs.range([50, 5000], \{value: 200, step: 10, label: "Tamaño total"\},),}
\NormalTok{  "Latencia A": Inputs.range([1, 50], \{value: 1, step: 0.1, label: "Latencia (A)"\}),}
\NormalTok{  "Throughput A": Inputs.range([1, 50], \{value: 10, step: 1, label:"Throughput (A)"\}),}
\NormalTok{  "Latencia B": Inputs.range([1, 50], \{value: 2, step: 0.1, label:"Latencia (B)"\}),}
\NormalTok{  "Throughput B": Inputs.range([1, 50], \{value: 20, step: 1, label:"Throughput B"\})}
\NormalTok{\})}

\NormalTok{// Extraemos variables del formulario}
\NormalTok{totalData = settings["Tamaño total"]}
\NormalTok{latencyA = settings["Latencia A"]}
\NormalTok{throughputA = settings["Throughput A"]}
\NormalTok{latencyB = settings["Latencia B"]}
\NormalTok{throughputB = settings["Throughput B"]}
\end{Highlighting}
\end{Shaded}

\begin{Shaded}
\begin{Highlighting}[]
\NormalTok{// Función progreso}
\NormalTok{function progress(t, latency, throughput, total) \{}
\NormalTok{  if (t \textless{} latency) return 0;}
\NormalTok{  return Math.min(100, ((t {-} latency) * throughput / total) * 100);}
\NormalTok{\}}

\NormalTok{// Reloj animado}
\NormalTok{time = \{}
\NormalTok{  let t0 = Date.now();}
\NormalTok{  while (true) \{}
\NormalTok{    yield (Date.now() {-} t0) / 100; // en decimas de segundo}
\NormalTok{    await Promises.tick(50);}
\NormalTok{  \}}
\NormalTok{\}}

\NormalTok{// Tiempo máximo}
\NormalTok{maxTime = Math.max(latencyA + totalData/throughputA, latencyB + totalData/throughputB)}

\NormalTok{// Progresos}
\NormalTok{progA = progress(time \% maxTime, latencyA, throughputA, totalData)}
\NormalTok{progB = progress(time \% maxTime, latencyB, throughputB, totalData)}

\NormalTok{// Gráfico con Plot}
\NormalTok{Plot.plot(\{}
\NormalTok{  y: \{domain: [0, 100], label: "\% completado"\},}
\NormalTok{  x: \{label: "Tiempo (s)"\},}
\NormalTok{  color: \{legend: true, domain: ["A", "B"], range: ["blue", "red"]\},}
\NormalTok{  marks: [}
\NormalTok{    Plot.line(d3.range(0, maxTime, 0.1).map(t =\textgreater{} (\{t, y: progress(t, latencyA, throughputA, totalData), series: "A"\})), \{x: "t", y: "y", stroke: "series"\}),}
\NormalTok{    Plot.line(d3.range(0, maxTime, 0.1).map(t =\textgreater{} (\{t, y: progress(t, latencyB, throughputB, totalData), series: "B"\})), \{x: "t", y: "y", stroke: "series"\}),}
\NormalTok{    Plot.dot([\{t: time \% maxTime, y: progA, series: "A"\}], \{x: "t", y: "y", fill: "series", r: 5\}),}
\NormalTok{    Plot.dot([\{t: time \% maxTime, y: progB, series: "B"\}], \{x: "t", y: "y", fill: "series", r: 5\})}
\NormalTok{  ]}
\NormalTok{\})}
\end{Highlighting}
\end{Shaded}
\end{frame}

\begin{frame}{Componentes de la Latencia}
\phantomsection\label{componentes-de-la-latencia}
\(d_{total} = \textcolor{red}{d_{proc}} + \textcolor{blue}{d_{queue}} + \textcolor{green}{d_{prop}} + \textcolor{orange}{d_{trans}}\)

\begin{itemize}
\tightlist
\item
  {dproc}: Procesamiento en router (microsegundos)
\item
  {dqueue}: Espera en buffer (variable con tráfico)
\item
  {dprop}: Propagación por el medio (d/s)
\item
  {dtrans}: Transmisión de datos (L/R)
\end{itemize}
\end{frame}

\begin{frame}{RTT}
\phantomsection\label{rtt}
\begin{quote}
RTT (Round trip time): Tiempo total que tarda un paquete en ir desde el
origen hasta el destino y volver de vuelta (ida + vuelta).
\end{quote}

\begin{itemize}
\tightlist
\item
  La latencia no tiene porque ser simétrica.
\item
  Generalmente la descarga es más rápida que la subida.
\item
  Por lo tanto, el RTT es un valor muy importante en aplicaciones
  interactivas.
\end{itemize}
\end{frame}

\begin{frame}{Comparación: Fibra vs 5G}
\phantomsection\label{comparaciuxf3n-fibra-vs-5g}
\begin{longtable}[]{@{}lll@{}}
\toprule\noalign{}
Factor & Fibra Óptica & 5G \\
\midrule\noalign{}
\endhead
\textbf{Propagación} & 67\% velocidad luz & 100\% velocidad luz \\
\textbf{Procesamiento} & \textasciitilde0.1ms/salto & \textasciitilde4ms
(estación radio) \\
\textbf{Cola} & Baja congestión & Alta congestión \\
\textbf{Transmisión} & Hasta 10 Gb/s & \textless{} 1 Gb/s \\
\bottomrule\noalign{}
\end{longtable}

\textbf{Resultado}: Fibra generalmente más rápida y estable
\end{frame}

\begin{frame}{Jitter: Variabilidad de Latencia}
\phantomsection\label{jitter-variabilidad-de-latencia}
\begin{quote}
Jitter: Variación en el tiempo de llegada de los paquetes que causa
inconsistencia en la comunicación.
\end{quote}

\begin{block}{Ejemplo comparativo}
\phantomsection\label{ejemplo-comparativo}
\begin{columns}[T]
\begin{column}{0.5\linewidth}
\textbf{Escenario 1} (Bajo jitter):

\begin{itemize}
\tightlist
\item
  Paquetes: 50, 52, 48, 51 ms
\item
  Promedio: 50.25 ms
\item
  Variación: 1.48 ms ✅
\end{itemize}
\end{column}

\begin{column}{0.5\linewidth}
\textbf{Escenario 2} (Alto jitter):

\begin{itemize}
\tightlist
\item
  Paquetes: 28, 68, 43, 62 ms
\item
  Promedio: 50.25 ms
\item
  Variación: 15.82 ms ❌
\end{itemize}
\end{column}
\end{columns}

\textbf{Impacto}: Voz entrecortada, saltos en video, degradación en
juegos
\end{block}
\end{frame}

\begin{frame}{Requisitos para Videojuegos}
\phantomsection\label{requisitos-para-videojuegos}
\begin{block}{RTT máximo tolerado}
\phantomsection\label{rtt-muxe1ximo-tolerado}
\begin{longtable}[]{@{}lll@{}}
\toprule\noalign{}
Género & Tolerancia & Ejemplo \\
\midrule\noalign{}
\endhead
\textbf{Fighting} & 16-50ms & Street Fighter \\
\textbf{FPS Competitivo} & 20-50ms & Counter-Strike \\
\textbf{Racing} & 50-100ms & Gran Turismo \\
\textbf{RTS} & 100-200ms & StarCraft \\
\textbf{MMORPG} & Variable & World of Warcraft \\
\textbf{Turn-based} & 500ms+ & Civilization \\
\bottomrule\noalign{}
\end{longtable}
\end{block}
\end{frame}

\begin{frame}{Pérdida de Paquetes}
\phantomsection\label{puxe9rdida-de-paquetes}
\begin{block}{Causas principales}
\phantomsection\label{causas-principales}
\begin{itemize}
\tightlist
\item
  \textbf{Congestión}: Buffers llenos en routers
\item
  \textbf{Corrupción}: Interferencias electromagnéticas
\item
  \textbf{Radiación cósmica}: \textasciitilde1 error/256MB/día
\end{itemize}
\end{block}

\begin{block}{Soluciones}
\phantomsection\label{soluciones}
\begin{itemize}
\tightlist
\item
  Protocolos de capas superiores (TCP)
\item
  Retransmisión automática
\item
  Códigos de corrección de errores
\item
  Interpolar la información
\end{itemize}
\end{block}
\end{frame}

\begin{frame}{Resumen}
\phantomsection\label{resumen}
\begin{itemize}[<+->]
\tightlist
\item
  Internet es un \textbf{sistema distribuido y descentralizado}
\item
  Evolución desde 4 hosts (1969) a \textgreater100B dispositivos (2025)
\item
  \textbf{Infraestructura jerárquica}: PAN → LAN → MAN → WAN → Internet
\item
  \textbf{Modelos de capas}: OSI (teórico) vs TCP/IP (práctico)
\item
  \textbf{Rendimiento}: Balance entre latencia y throughput
\item
  \textbf{Aplicaciones críticas}: Videojuegos requieren \textless50ms
  para competitivo
\end{itemize}
\end{frame}

\end{document}
